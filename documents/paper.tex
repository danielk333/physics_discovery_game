\documentclass{article}
\usepackage[utf8]{inputenc}

\title{Numerical simulation and modeling to promote expert-like behaviour and competence in applied physics among students}
\author{Daniel Kastinen}
\date{January 2021}

\begin{document}

\maketitle

ok here is the entire idea

introducing physics as observations and modeling!

form groups of 2-3

game played by groups computer.

game is:
A blip in space with unexplained physics, goal = get to portal on other side
you have a control system for your "ship"
the control system is based on a physics model
implement correct "physics" model in class method and implement "control system" to get to goal

GAME: ship has a sensor, generates a file with positions and velocities as a function of time, and fuel

stage 1: control system disabled - Probe, experiment, discover the dynamics, build mathematical models
stage 2: run "simulations", assuming the models are correct, to design control system
stage 3: run actual mission

finish level 1, get to level 2 (same, should complete automatically),
level 3 (new physics appear)

use level 3 to discuss limitations of models

THEY JUST EXPERIENCED DISCOVERING AND APPLYING PHYSICS

We can never have perfect model because we can never try it in infinite conditions, e.g. speed of light might vary with universe expansion or age

Write a report on how the physics rules derived and their consequences for the "universe" of the ship and the success/design of the control system
THEY SHOULD KEEP A LOGBOOK, on every step they perform

review all solutions in the end in front of class
all same start or all different


- 

\section{Introduction}

\section{Education Component}

\subsection{Suggested course context}


\subsection{Teaching and Learning goals}

Intended Learning Outcomes (ILOs)

\subsection{Teaching and Learning Plan}


\subsection{Material}
actual simulation code

\subsection{Execution}

computer lab room sessions 

attachement
1 - lab instructions

~2ECTS of work?

\subsection{Assessment}

written report <- summative
continues logbook by groups <- formative
end-of-day summary on ideas and progress from each group / short discussion (teacher note down progress while listening) <- formative

\section{Evaluation}

\subsection{Pre- and post-test}

we use CLASS with 5 modifications

v3 has a few "not scored" items, we have changed these to serve our purpuese

\subsection{Interviews}

\subsection{Educational variables}

things that can easily be changed in light of pre/post test and interview results.

\section{Results}

\section{Discussion}

\end{document}
